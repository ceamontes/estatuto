\documentclass{estatuto}

\begin{document}

	\titulo{Centro de Estudos Astronômicos e Ciências de Montes Claros -- CEAMONTES}

	\begin{artigos}
		\section{Da denominação, sede e fins}
			\item A Associação Centro de Estudos Astronômicos e Ciências de Montes Claros, também designada pela sigla CEAMONTES, fundada em \red{16 de outubro de 2019}, é uma associação, sem fins econômicos, que terá duração por tempo indeterminado, sede no Município de Montes Claros, Estado de Minas Gerais, na rua (avenida) ..................... Bairro ......... e foro em Montes Claros-MG.
			\item A Associação tem por finalidades:
				\begin{itens}
					\item Congregar os astrônomos e cientistas da cidade e região;
					\item Zelar pela liberdade de ensino e pesquisa;
					\item Zelar pelos interesses e direitos dos astrônomos e cientistas;
					\item Zelar pelo prestígio da ciência na região norte-mineira;
					\item Estimular as pesquisas e o ensino da ciência na região norte-mineira;
					\item Manter contato com institutos e sociedades correlatas no País e no exterior;
					\item Promover reuniões científicas, congressos especializados, cursos e conferências;
					\item Editar informativos sobre as atividades do CEAMONTES e assuntos gerais relacionados com seus objetivos sociais;
					\item Estabelecer redes, parcerias e intercâmbios com sociedades científicas, instituições de ensino e/ou pesquisa, organizações não governamentais, universidades, Poder Público e outras entidades, públicas ou privadas, nacionais ou estrangeiras; e
					\item Promover campanhas de mobilização de recursos para financiar programas e projetos, próprios ou de terceiros.
				\end{itens}
			\item No desenvolvimento de suas atividades, a Associação não fará qualquer discriminação de raça, cor, sexo ou religião.
			\item A Associação poderá ter um Regimento Interno, que aprovado pela Assembleia Geral, disciplinará o seu funcionamento.
			\item A fim de cumprir suas finalidades, a Associação poderá organizar-se em tantos Núcleos, quantos se fizerem necessários, os quais se regerão pelo Regimento Interno.
			
		\section{Dos associados}
			\item A Associação é constituída por número ilimitado de associados, que serão admitidos, a juízo da diretoria, dentre pessoas idôneas.
			\item Haverá as seguintes categorias de associados:
				\begin{itens}
					\item Fundadores, os que assinarem a ata de fundação da Associação;
					\item Beneméritos, aqueles aos quais a Assembleia Geral conferir esta distinção, espontaneamente ou por proposta da diretoria, em virtude dos relevantes serviços prestados à Associação;
					\item Honorários, aqueles que se fizerem credores dessa homenagem por serviços de notoriedade prestados à Associação, por proposta da diretoria à Assembleia Geral;
					\item Contribuintes, os que pagarem a mensalidade estabelecida pela Diretoria; e
					\item Juniores, destinados a estudantes de nível fundamental, médio e superior, menores de 18 anos, sujeitos a condições específicas de participação e isentos de contribuição mensal.
				\end{itens}
			\item São direitos dos associados quites com suas obrigações sociais, excetuado os juniores:
				\begin{itens}
					\item votar e ser votado para os cargos eletivos; e
					\item tomar parte nas assembleias gerais.
				\end{itens}
				\paragrafo{Os associados beneméritos e honorários não terão direito a voto e nem poderão ser votados.}
			\item São deveres dos associados:
				\begin{itens}
					\item cumprir as disposições estatutárias e regimentais;
					\item acatar as determinações da Diretoria; e
					\item arcar com o pagamento das contribuições mensais determinadas pela diretoria, exceto se alcançar a condição de remido, após 30 anos de contribuição ininterrupta e 60 anos de idade.
				\end{itens}
				\paragrafo{Havendo justa causa, o associado poderá ser excluído da Associação por decisão da diretoria, após o exercício do direito de defesa. Da decisão caberá recurso à Assembleia Geral.}
			\item Os associados da entidade não respondem, nem mesmo subsidiariamente, pelas obrigações e encargos sociais da instituição.
			\item Poderão se a associar à entidade as pessoas naturais ou jurídicas relacionadas ao ideal e aos valores da associação.
			\begin{paragrafos}
				\item Será excluído o associado que:
					\begin{itens}
						\item deixar de cumprir sua obrigação estatutária para com a associação;
						\item praticar atos infringindo o previsto na lei, no estatuto ou regimento interno; e
						\item não cumprir, sem justificativa, as resoluções oriundas da diretoria executiva.
					\end{itens}
				\item A decisão da diretoria será comunicada ao interessado no prazo de cinco dias úteis, após regular processo disciplinar.
			\end{paragrafos}
		\section{Da administração}
			\item A Associação será administrada por:
				\begin{itens}
					\item Assembleia Geral;
					\item Diretoria Executiva; e
					\item Conselho Fiscal.
				\end{itens}
			\item A Assembleia Geral, órgão soberano da instituição, constituir-se-á dos associados em pleno gozo de seus direitos civis e estatutários.
			\item Compete à Assembleia Geral:
				\begin{itens}
					\item eleger a Diretoria e o Conselho Fiscal;
					\item destituir os administradores;
					\item apreciar recursos contra decisões da Diretoria Executiva;
					\item decidir sobre reformas do Estatuto;
					\item conceder o título de associado benemérito e honorário por proposta da Diretoria Executiva;
					\item decidir sobre a conveniência de alienar, transigir, hipotecar ou permutar bens patrimoniais;
					\item decidir sobre a extinção da entidade;
					\item aprovar as contas; e
					\item aprovar o regimento interno.
				\end{itens}
			\item A Assembleia Geral realizar-se-á, ordinariamente, uma vez por ano para:
				\begin{itens}
					\item apreciar o relatório anual da Diretoria Executiva; e
					\item discutir e homologar as contas e o balanço aprovado pelo Conselho Fiscal.
				\end{itens}
				\paragrafo{A Assembleia Geral poderá ser realizada de forma eletrônica, desde que tal modalidade conste no edital de convocação.}
			\item A Assembleia Geral realizar-se-á, extraordinariamente, quando convocada:
				\begin{itens}
					\item pelo presidente da Diretoria Executiva;
					\item por maioria da Diretoria Executiva;
					\item pelo Conselho Fiscal; e
					\item por requerimento de 1/5 dos associados quites com as obrigações sociais.
				\end{itens}
			\item A convocação da Assembleia Geral será feita por meio de edital afixado na sede da Instituição e enviado por qualquer meio eletrônico, com antecedência mínima de 10 dias.
				\paragrafo{Qualquer Assembleia instalar-se-á em primeira convocação com a maioria dos associados e, em segunda convocação, com qualquer número, não exigindo a lei quorum especial.}
			\item A Diretoria Executiva será constituída por um Presidente, um Vice-Presidente, Secretário e Tesoureiro.
				\paragrafo{O mandato da Diretoria Executiva será de (4) quatro anos, podendo ser reeleita, excetuando-se o cargo de Presidente.}
			\item Compete à Diretoria Executiva:
				\begin{itens}
					\item elaborar e executar programa anual de atividades;
					\item elaborar e apresentar, à Assembleia Geral, o relatório anual;
					\item estabelecer o valor da mensalidade para os sócios contribuintes;
					\item entrosar-se com instituições públicas e privadas para mútua colaboração em atividades de interesse comum;
					\item contratar e demitir funcionários;
					\item Instituir e extinguir Núcleos; e
					\item convocar a Assembleia Geral.
				\end{itens}
			\item A Diretoria Executiva reunir-se-á no mínimo semestralmente.
			\item Compete ao Presidente:
				\begin{itens}
					\item representar a Associação ativa e passivamente, judicial e extrajudicialmente;
					\item cumprir e fazer cumprir este Estatuto e o Regimento Interno;
					\item convocar e presidir a Assembleia Geral:
					\item convocar e presidir as reuniões da Diretoria Executiva; e
					\item assinar, com o tesoureiro, todos os cheques, ordens de pagamento e títulos que representem obrigações financeiras da Associação.
				\end{itens}
			\item Compete ao Vice-Presidente:
				\begin{itens}
					\item substituir o Presidente e demais cargos, inclusive o tesoureiro, em suas faltas ou impedimentos;
					\item assumir o mandato, em caso de vacância, até o seu término; e
					\item prestar, de modo geral, a sua colaboração ao Presidente.
				\end{itens}
			\item Compete ao Secretário:
				\begin{itens}
					\item secretariar as reuniões da Diretoria e Assembleia Geral e redigir e assinar as atas; e
					\item publicar todas as notícias das atividades da entidade.
				\end{itens}
			\item Compete ao Tesoureiro:
				\begin{itens}
					\item arrecadar e contabilizar as contribuições dos associados, rendas, auxílios e donativos, mantendo em dia a escrituração;
					\item pagar as contas autorizadas pelo Presidente;
					\item apresentar relatórios de receita e despesas, sempre que forem solicitados pelo Presidente ou maioria da Diretoria Executiva;
					\item apresentar o relatório financeiro para ser submetido à Assembleia Geral;
					\item apresentar semestralmente o balancete ao Conselho Fiscal;
					\item conservar, sob sua guarda e responsabilidade, os documentos relativos à tesouraria;
					\item manter todo o numerário em estabelecimento de crédito; e
					\item assinar, com o presidente, todos os cheques, ordens de pagamento e títulos que representem obrigações financeiras da Associação.
				\end{itens}
			\item O Conselho Fiscal será constituído por (3) três membros, eleitos pela Assembleia Geral.
				\begin{paragrafos}
					\item O mandato do Conselho Fiscal será coincidente com o mandato da Diretoria Executiva.
					\item Em caso de vacância de maioria do conselho, far-se-á nova eleição das vagas remanescentes por Assembleia Geral extraordinária.
				\end{paragrafos}
			\item Compete ao Conselho Fiscal:
				\begin{itens}
					\item examinar os livros de escrituração da entidade;
					\item examinar os balancetes apresentados pelo Tesoureiro, opinando a respeito; e
					\item opinar sobre a aquisição e alienação de bens.
				\end{itens}
				\paragrafo{O Conselho Fiscal reunir-se-á ordinariamente a cada (6) seis meses e, extraordinariamente, sempre que necessário.}
			\item As atividades dos diretores e conselheiros, bem como as dos associados, serão inteiramente gratuitas, sendo-lhes vedado o recebimento de qualquer lucro, gratificação, bonificação ou vantagem.
			\item A instituição não distribuirá lucros, resultados, dividendos, bonificações, participações ou parcela de seu patrimônio, sob nenhuma forma ou pretexto.
			\item A Associação manter-se-á através de contribuições dos associados e de outras atividades, sendo que essas rendas, recursos e eventual resultado operacional serão aplicados integralmente na manutenção e desenvolvimento dos objetivos institucionais.
		\section{Do patrimônio}
			\item O patrimônio da Associação será constituído por:
				\begin{itens}
					\item contribuições, rendas eventuais, doações e legados;
					\item bens móveis, imóveis, veículos, semoventes, ações e apólices de dívida pública; e
					\item subvenção e auxílio estabelecido pelo poder público.
				\end{itens}
				\begin{paragrafos}
					\item Nenhum bem da associação será alienado sem aprovação da Diretoria Executiva, com prévio parecer do Conselho Fiscal.
					\item No caso do parágrafo anterior, o produto da venda será aplicado na aquisição de outros bens ou na realização estrita dos objetivos da associação.
				\end{paragrafos}
			\item No caso de dissolução da Instituição, o patrimônio remanescente será destinado a outra instituição congênere, com personalidade jurídica ou entidade Pública.
		\section{Das disposições gerais}
			\item A Associação será dissolvida por decisão da Assembleia Geral Extraordinária, especialmente convocada para esse fim, quando se tornar impossível a continuação de suas atividades.
			\item Os casos omissos serão resolvidos pela Diretoria Executiva e referendados pela Assembleia Geral.
	\end{artigos}

	\assinaturas
\end{document}